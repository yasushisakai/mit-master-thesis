%% This is an example first chapter.  You should put chapter/appendix that you
%% write into a separate file, and add a line \include{yourfilename} to
%% main.tex, where `yourfilename.tex' is the name of the chapter/appendix file.
%% You can process specific files by typing their names in at the 
%% \files=
%% prompt when you run the file main.tex through LaTeX.
\chapter{5. Disscussion}

\section{Framework of collective urban design}
modular building blocks of participatory urban design
scale -> goes back to how we define "cities", autonomous units of direct democracy
[diagram]
definition of data collection, different modes of micro participation
sensors as humans?
a mapping of one existing situation (often unstructured / hard to grasp) to another quantifiable form (number)
data from sensors
data from humans, low level to high level (fuzziness, level of integration)
reporting data (low level)
reporting impression
proposing for an intervention (high level, fuzzy)
\subsection{a general platform to conduct participatory urban design.}
Current applications like citizens connect is a general purpose app limiting at reporting. 

While it will be useful to have a general app that goes beyond reporting but enables users to propose an intervention, there is a risk that it is hard to maintain the attention for one mobile app. An alternative is to create a framework of the app, which can be customized to specific targets similar to bike lane security and topics discussed in the DISCUSSION section.

Recent web application development have focused on modularising parts to make it easy to reuse and distribute. We can observe this by NPM (Node Package Manager; A software that automates dependency checking and installing parts of software to be used in combination.) Having more than 475,000\footnote{https://www.npmjs.com/} building blocks that one can combine to make web based applications. The elements of one application is modularising as well, were one application are components combined together.

\subsection{Communal value to Civic value}
Clay Shirkey points out in “Cognitive Surplus.” The internet has lowered the threshold barrier to coordinating with each other, and within this collective action, there are two kinds of value mechanisms; which are Communal and Civic values. Communal value is created by and acknowledged the same group. Although bikebump for urban issues, but is fundamentally for bikers by bikers, which makes bikebump's engine the communal value behavior. Allowing others using different modes of transportation is the next step for inclusion.

\subsection{How to tackle "relative" valuing}
Multiple people have commented the relativeness of their perception. Data shows that they feel insecure when bike lanes disappear. Using a gradational input like a volume knob will be suitable for getting a continuous input. The users feedback also cast new questions on how to annotate the city. Bikebump used a 15m radius geofence as a method, but users implied that 'good' situations are likely to be a continuous experience rather than a pulsive reaction. Looking at a constant input may also validate this perception.

\section{Interaction between users}
Social proofing
\section{Micro owning the city}
Each report as declaration of ownership = share economy
\section{Scope and different application}
pedestrian sidewalks
street lights
regional developments

